\documentclass{article}

	\author{王贺,2014060102018}
	\title{2015.10.12,数据结构与算法作业1}
\usepackage{ctex}
\usepackage{minted}
\usepackage{clrscode}



\begin{document}
	\maketitle
\section{Question 1} 

将两个递增的有序链表合并为一个递增的有序链表。要求结果链表仍使用原来两个链表的存储空间,不另外占用其他的存储空间。表中不允许有重复的数据。

\subsection{语言描述} 

ListA,ListB合并到ListC\par

1)选取ListA和ListB第一个数据较小的添加到ListC,将较小的那个指向后继后删除路过的结点\par

2)比较ListA和ListB当前数据,将较小的添加到ListC中,如果出现相同的则删除其一,直至链表尾\par

3)将当前不为空的后继所有结点依次添加到ListC中

\subsection{伪代码描述}



\begin{codebox}
\Procname{$\proc{mergeList(A)}$}
\li $la \gets ListA$
\li $lb \gets ListB$
\li $lc \gets ListC$
\li \If  $la.data < lb.data$				 
\li 	\Then
\li			$lc \gets la$
\li 		$la \gets del(la)$
\li 	\Else
\li     	$lc \gets lb$
\li 		$lb \gets del(lb)$
	    \End
\li \While $la.next \neq null\ \  and\ \  lb.next \neq null $  
\li 	\Do
\li	 	\If $ la.data = lb.data $  
\li 		\Then 
\li 			$ append(lc, la, True) $
\li 		\Else
\li 			\If $ la.data < lb.data $
\li					\Then
\li 					$ append(lc, la, False) $
\li 				\Else
\li 					 $ append(lc, lb, False) $
	 			\End
	  	\End
	 	\End
\li  	\If $ la.next = NULL $
\li 		\Then 
\li  			$ lc.next \gets lb $
\li 		\Else
\li			$ lc.nect \gets la $
	 	\End
		

\end{codebox}



\subsection{代码描述}

\begin{minted}[mathescape,
               linenos,
               numbersep=5pt,
               gobble=2,
               frame=lines,
               framesep=2mm]{csharp}
    typedef struct
    {
        int data;
        struct node* next;
    }*LinkNode, Linklist;


    LinkNode del(LinkNode  l)
    {
        LinkNode temp = l;
        l = l ->next;
        //delete temp;
        return l;
    }

    LinkNode append(LinkNode front, LinkNode next, erase = Flase)
    {	
    
        LinkNode temp = front -> next;           //save front's next
        front -> next = next;           //insert
        front -> next -> next = temp;   //update the insert
        next = next -> next;            //update next
        if (erase)                       //delete
        {
            del(next);
        }
    }

    void mergeList(LinkList ListA, LinkList ListB, LinkList &ListC)
    {
        LinkNode la,lb,lc;
        la = ListA;
        lb = ListB;
        lc = ListC;
        la -> data < lb -> data ? {lc = la; la = del(la);} : {lc = lb; lb = del(lb);};

        while (la -> next != NULL && lb -> next != NULL)
        {
            if (la -> data == lb -> data)
            {
                append(lc, la, True);
            }
            else
            {
                la -> data < lb -> data ? append(lc, la, False) : append(lc, lb, False);
            }
        }
        la -> next == NULL ? lc -> next = lb : lc -> next = la;
    }




\end{minted}










\section{Question 2} 

将两个非递减的有序链表合并为一个非递增的有序链表。要求结果链表仍使用原来两个链表的存储空间,不另外占用其他的存储空间。表中允许有重复的数据。


\subsection{语言描述} 

1)以ListA,ListB小的元素结点为跟踪标识,并定义扫描标识\par
2)扫描标识数据位比较,取小的,并将取走后的标识位后移(删除无用的头结点)\par
3)更新上一步骤取走的结点的后继为跟踪标识,更新跟踪标识为这个结点\par
4)回到步骤2直至一链表为空,则重复3填充跟踪标识

\subsection{伪代码描述}



\begin{codebox}
\Procname{$\proc{mergeList(B)}$}

\li \If $ ListA.data <= ListB.data $
\li 	\Then 
\li 		$ li \gets ListA$
\li 		$ scannerA \gets ListA.next $
\li 		$ scannerB \gets ListB $
\li 		\Comment if delete 
\li 		$ delete ListA$		
\li 	\Else
\li 		$ li \gets ListB $
\li 		$ scannerB \gets ListB.next $
\li 		$ scannerA \gets ListA $
\li 		\Comment if delete 
\li 		$ delete ListB$		
	\End
\li \While $ scannerA \neq NULL\ \ and \ \ scannerB \neq NULL $
\li 	\Do
\li 	\If $ scannerA.data <= scannerB.data $
\li 		\Then 
\li 			$ li \gets li.next \gets scannerA $
\li 			$ scannerA \gets scannerA.next $
\li 		\Else
\li 			$ li \gets li.next \gets scannerA  $
\li 			$ scannerB \gets scannerB.next $
		\End
	\End
\li \If $ scannerA = NULL $
\li 	\Then 
\li 		$ scanner \gets scannerB $
\li 	\Else
\li 		$ scanner \gets scannerA $
	\End	
\li \While $ scanner \neq NULL $
\li 	\Do
\li 		$ li \gets li.next \gets scanner $
\li 		$ scanner \gets scanner.next $
\end{codebox}



\subsection{代码描述}

\begin{minted}[mathescape,
               linenos,
               numbersep=5pt,
               gobble=2,
               frame=lines,
               framesep=2mm]{csharp}
    void mergeList(LinkList &ListA, LinkList &ListB)
    {
        LinkNode scannerA,scannerB,li;
        scannerA = scannerB = li = NULL;

        if (ListA -> data <= ListB -> data)
        {
            li = ListA;
            scannerA = ListA -> next;
            scannerB = ListB;
            //delete ListA;
        }
        else
        {
            li = ListB;
            scannerB = ListB -> next;
            scannerA = ListA;
            //delete ListB;
        }

        while (scannerA != NULL && scannerB != NULL)
        {
            if (scannerA -> data <= scannerB -> data)
            {
                li = li -> next = scannerA;
                scannerA = scannerA -> next;
            }
            else
            {
                li = li -> next = scannerB;
                scannerB = scannerB -> next;
            }
        }
        LinkNode scanner = NULL;
        scanner = scannerA == NULL ? scannerB : scannerA;
        while (scanner != NULL)
        {
            li = li -> next = scanner;
            scanner = scanner -> next;
        }
    }

\end{minted}

\section{Question 3} 

设计一个算法,通过一趟遍历在单链表中确定值最大的结点。


\subsection{语言描述} 

1)跟踪标识为头数据结点,扫描标识为下一个结点\par
2)比较两标识数据,如果跟踪标识小,则移动跟踪标识;否则扫描标识后移\par
3)当扫描标识为空,跟踪标识为最大

\subsection{伪代码描述}
\begin{codebox}
\Procname{$\proc{findMax}$}

\li $maxNode \gets list$
\li $ cursor \gets list.next$
\li \While $ cursor \neq NULL $
\li 	\Do
\li 	\If $ maxNode.data < cursor.data $
\li 		\Then 
\li 			$ maxNode \gets cursor $
		\End
\li $ cursor \gets cursor.next $  

\end{codebox}



\subsection{代码描述}

\begin{minted}[mathescape,
               linenos,
               numbersep=5pt,
               gobble=2,
               frame=lines,
               framesep=2mm]{csharp}
    void findMax(LinkList list)
    {
        LinkNode maxNode = list;
        LinkNode cursor = list -> next;

        while(cursor != NULL)
        {
            if (maxNode -> data < cursor -> data)
            {
                maxNode = cursor;
            }
            cursor = cursor -> next;
        }
    }

\end{minted}

\section{Question 4} 

设计一个算法,通过一趟遍历,将链表中所有结点的链接方向逆转,且仍利用原表的存储空间。


\subsection{语言描述} 

1)跟踪结点为头结点,保存跟踪结点的后继为扫描结点,将头结点后继指向null,跟踪结点更新到扫描结点\par
2)保存跟踪结点的后继为扫描结点,将跟踪结点后继指向跟踪结点本身,跟踪结点更新到扫描结点\par

\subsection{伪代码描述}



\begin{codebox}
\Procname{$\proc{reverse}$}
\li $ tracker \gets list $
\li $ cursor \gets list.next $
\li $ list.next \gets NULL$
\li \While $ cursor \neq NULL $
\li 	\Do
\li 		$ cursor \gets tracker.next $
\li 	 	$ tracker.next \gets tracker $
\li 		$ tracker \gets cursor $
	\End
\li \Return $ tracker $ 

\end{codebox}



\subsection{代码描述}

\begin{minted}[mathescape,
               linenos,
               numbersep=5pt,
               gobble=2,
               frame=lines,
               framesep=2mm]{csharp}
    LinkNode reverse(LinkList& list)
    {
        LinkNode tracker = list;
        LinkNode cursor = list -> next;
        list -> next = NULL;
        while (cursor != NULL)
        {
            cursor = tracker -> next;
            tracker -> next = tracker;
            tracker = cursor;
        }
        return tracker;
    }

\end{minted}

\section{Question 5} 


将编号为0和1的两个栈存放于一个数组空间V[m]中,栈底分别处于数组的两端。当第0号栈的栈顶指针top[0]等于-1时该栈为空;当第1号栈的栈顶指针top[1]等于m时,该栈为空。两个栈均从两端向中间增长。试编写双栈初始化,判断栈空、栈满、进栈和出栈等算法的函数。双栈数据结构的定义如下:\par
\begin{minted}[mathescape,
               linenos,
               numbersep=5pt,
               gobble=2,
               frame=lines,
               framesep=2mm]{csharp}
    typedef struct{
        int top[2], bot[2];  //栈顶和栈底指针
        SElemType *V;        //栈数组 
        int m;               //栈最大可容纳元素个数
    }DblStack;
\end{minted}


\subsection{代码描述}

\begin{minted}[mathescape,
               linenos,
               numbersep=5pt,
               gobble=2,
               frame=lines,
               framesep=2mm]{csharp}
    typedef struct
    {
        int top[2], bot[2];  //栈顶和栈底指针
        SElemType *V;        //栈数组 
        int m;              //栈最大可容纳元素个数
    }DblStack;

    void initDblStack(DblStack& s, SElemType size)
    {
        s.m = size;
        s.V = new SElemType(size);
        s.top[0] = s.bot[0] = -1;
        s.top[1] = s.bot[1] = size -1;
    }



    int IsEmpty(DblStack s, int i)
    {
        if (i == 0)
        {
            return s.top[0] == -1 ? 1 : 0;
        }
        else if (i == 1)
        {
            return s.top[1] == s.m ? 1 : 0;
        }
        else
        {
            return -1;
        }
    }

    int IsFull(DblStack s)
    {
        return s.top[1]-s.top[0] > 0 ? 0 : 1;
    }

    void Dblpush(DblStack &s,SElemType x,int i)
    {
        if (i == 0 && s.top[0] + 1 < s.top[1])
        {
           s.V[++s.top[0]] = x;
        }
        else if (i == 1 && s.top[1] - 1 > s.top[0])
        {
            s.V[--s.top[1]] = x;
        }

    }

    int Dblpop(DblStack &s,int i,SElemType &x)
    {
        if (i == 0 && !IsEmpty(s, i))
        {
            x = s.V[top[0]--];
        }
        else if (i == 1 && !IsEmpty(s, i))
        {
            x = s.V[top[1]++];
        }
        else
        {
            return 0;
        }
        return 1;
    }

\end{minted}


\end{document}